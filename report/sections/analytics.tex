\section{Analytics}
As we have previously mentioned, in our visualization play a fundamental role the interactions between the available charts.
In fact, the interactions transform a static visualization into a dynamic one, that is what we humans are used to see.
For space reasons, we will represent with a $*$ all available charts in our project.
In particular, the interactions provided are the following:
\begin{itemize}
    \item \textbf{geographic map} $\rightarrow *$: there is a possibility to select one or multiple battles through brushing technique. In particular, when a window is drawn, we can leverage its boundary in order to trigger a run time computation to exclude (i.e. hide) the battles outside the brush area and show only the ones inside it.At the end of this task, another run time function is triggered to update all the remaining charts, constraining their visualization domain in order to show only the selected battles.
    \item \textbf{line chart} $\rightarrow$ \textbf{geographic map, stacked bar chart, scatter plot}: in this case we used brushing and zooming techniques for a possible interaction. As before, with the brushing we are able to understand the subset of battles that we want to show and with zooming we are able to update axis's ranges and data. To make things clear, the zoom area is exactly the same of the brushing one. Once that this update is performed, a run time function is called to update the data shown in the geographic map, in the stacked bar chart and in the scatter plot.
    \item \textbf{stacked bar chart} $\rightarrow$ null: this chart does not provide any interaction, it is only updated by others.
    \item \textbf{box plot} $\rightarrow$ null: this chart does not provide any interaction as well, but it is updated by others.
    \item \textbf{scatter plot} $\rightarrow$ \textbf{geographic map}: in this chart we have used the brushing approach to select a group of battles that will be highlighted with a dark border in case of light theme (resp. with a white border in case of dark theme). This action also triggers another run time function, which is responsible to highlight the selected battles also in the geographic map.
\end{itemize}

The analytics part is constituted by two main functions:
\begin{itemize}
    \item \textbf{Percentage of ground / naval / civil / not civil battles}: at run time will be triggered a function that computes the exact amount of each type of battles reported above, that will be displayed nearby the corresponding battle type label.
    \item \textbf{Box plot quantiles}: they are the principal components of this chart, which are computed at run time.
\end{itemize}
